\documentclass[12pt]{article}
\usepackage[utf8]{inputenc}
\usepackage[russian]{babel}

\title{Домашняя работа №1}
\author{Юшина Екатерина}
\date{}

\renewcommand{\tg}{\tan}

\begin{document}
	\maketitle
	\begin{flushright}
	    \textit{Audi multa, \\loquere pauca}
	\end{flushright}
	\par
	\vspace{20pt}
	Это мой первый документ в системе компьютерной верстки \LaTeX \\
	\begin{center}
	    {\Huge\textsf{<<Ура!!!>>}}
	\end{center}
    \par
	А теперь формулы. \textsc{формула}~--- краткое и точное словестное выражение, определение или же ряд математических величин, выраженный условными знаками.\\[15pt]
	
	\hspace{28pt} {\LARGE\textbf{Термодинамика}} \par
	Уравнение Менделеева--Клайперона~--- уравнение состояния идеального газа, имеющее вид $ pV = \nu RT $, где $p$~--- давление, $V$~--- объем, занимаемый газом, $T$~--- температура, $\nu$~--- колличество вещества газа, а $R$~--- универсальная газовая постоянная.\\[15pt]
	
	\hspace{28pt} {\LARGE\textbf{Геометрия \hfillПланеметрия}} \par
	Для \textsl{плоского} треугольника со сторонами $a$, $b$, $c$ и углом $\alpha$, лежащим против стороны $a$, справедливо соотношение
	$$
	    a^2 = b^2 + c^2 - 2ab \cos\alpha,
	$$
	из которого можно выразить косинус угла треугольника:
	$$
	    \cos\alpha = \frac{b^2 + c^2 - a^2}{2ab}.
	$$ \par
	Пусть $p$~--- полупериметр треугольника, тогда путем несложных преобразований можем получить, что
	$$
	    \tg\frac{\alpha}{2} = \sqrt{\frac{(p - b)(p - c)}{p(p - a)}}.
	$$
	\\[1cm] 
	На сегодня, пожалуй, хватит\dots Удачи!
\end{document}
